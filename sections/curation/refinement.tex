\subsection{Genotype refinement and phasing}
We carry out genotype refinement and phasing with SHAPEIT2.\cite{Delaneau2012} Initial refinement of genotype likelihoods is however carried out with Beagle4.\cite{Browning20071084} The posterior probabilities calculated by Beagle4 are then used as input for SHAPEIT2 along with a haplotype scaffold generated from SNP array data available for the same populations. The SNP array data undergoes QC per population and is phased with SHAPEIT2 across all populations to generate the haplotype scaffold. MVNcall can also utilize a haplotype scaffold and unlike SHAPEIT2 works for multiallelic sites.\cite{Menelaou2013} However, we use SHAPEIT2, because it improves on the accuracy of phasing.\cite{2014Delaneau} The Illumina Omni2.5M SNP array has been shown to be an optimal/sufficient haplotype scaffold size in African populations.\cite{Menelaou2013}\cite{2014Delaneau} Pedigree information will be used by SHAPEIT2 when available. For Beagle4 this is less important, as this is only an initial refinement used as input for SHAPEIT2. We use the duoHMM method of SHAPEIT2 for phasing, because it has been shown to have a lower switch error rate, when pedigree information is available.\cite{OConnell2014}
%Figure 8: Overall percent genotype discordance for different scaffold SNP density. Results are presented by population groups: African, European and Asian.
%Is MVNCall only better, if BEAGLE doesn't have phasing information?
%2014Delaneau - "It has been observed that the Beagle method does not have this property, and that Thunder and Impute2 benefit from using an initial set of haplotypes estimated via Beagle."
%2014Delaneu - "This approach generalizes our MVNcall, approach which is designed to phase one variant site at a time onto a haplotype scaffold, and improves upon its accuracy, by phasing multiple sites jointly onto the scaffold and using a more sophisticated underlying model."

\subsection{Quality control after variant calling and genotype refinement}
If SNP array data is available, we further check after genotype refinement with Beagle4 that the concordance between chip and sequence genotypes is greater than 0.98 for each sample. We also check for heterozygosity outliers (\textgreater3SD) and PCA outliers after genotype refinement. Variant calling and genotype refinement is repeated if a sample either fails the check of genotype concordance, which is a sign of poor data quality, or fails the check of heterozygosity, which can be a sign of sample contamination or a population outlier.