\section{Curation}
Given the need to include large sample sizes from diverse populations for such a resource for development of an African chip, curation of data will involve generation of homogenised data from different datasets, including publicly available data as outlined in Table 1. Given the heterogeneity in sequencing and coverage among datasets (Table 1), these will need to be processed in subsets, and then merged to generate a complete panel of all variants. 

Publicly available curated data, such as from the \href{http://www.1000genomes.org}{1000 Genomes project}, HGDP and the \href{http://www.simonsfoundation.org/}{Simon’s Foundation} will be used as such. For datasets that are sequenced in house, consistent methods will be used for curation.

\subsection{Alignment and preprocessing of reads}
Following generation of raw reads mapping will be carried out to the human reference genome that is current at the time (GRCh37/GRCh38) using BWA/BWA-mem. Optical and PCR duplicate reads will be marked with Picard MarkDuplicates on a lanelet level. The reads will be sorted by coordinate with samtools sort. The lanelets will be merged to a library level with Picard MergeSamFiles, and duplicate reads will be marked on a library level. The BAMs will be merged to a sample level and then sample level bam improvement will be carried out using GATKv2.8+. This process will consist of a per-sample realignment of reads around known and discovered indels with GATK RealignerTargetCreator and IndelRealigner in addition to base quality score recalibration (BQSR) with GATK BaseRecalibrator and PrintReads.

\subsection{QC}
We require that the percentage of aligned reads is 90\% or greater, which we check with samtools flagstat. Prior to variant calling we check the gender of the samples with GATK3.3+ DepthOfCoverage and require that the ratio between the non-PAR X and Y coverage is less than 2 and greater than 5 for males and females, respectively. Prior to variant calling we check for contamination and sample mix up with VerifyBamID and require that the calculated FREEMIX is less than 0.05. If SNP array data is available, we further check after variant calling that the concordance between chip and sequence genotypes is greater than 0.98 for each sample. We check for heterozygosity outliers (\textgreater3SD) and PCA outliers after variant calling.

\subsection{Variant calling}
One sample from the Genome in a Bottle (GiaB) highly curated set will be included for validation of the data processing pipeline (NA12878)\cite{Zook2014} (http://genomeinabottle.org) in each subset. PCR-free reads will be used for these validation samples, to avoid PCR artefacts. The validation sample will be downsampled to the coverage of the dataset being called, and processed through the same pipeline, to provide a comparator against a validated highly curated set of variants for this sample. The GiaB sample (NA12878) represents a sample from a 12 person pedigree that has gone through extensive curation and validation of variants.\cite{Zook2014} The accuracy of called data from this sample will provide a guide to the accuracy of the workflow applied. After applying standard QC filtering to bams, variant calling will be carried out using several algorithms (Samtools, Haplotype caller and Unified Genotyper). Calling of samples will be carried out with the GiaB genome sample, in order to assess the sensitivity and specificity of any calling pipeline for SNPs and structural variants. The pipeline with the greatest area under the ROC (sensitivity, 1-specificity) curve will be used for calling variants. These callers may be different for SNPs, short indels and SVs. Calling and filtering of SNPs, indels and long deletions will be carried out separately using the chosen algorithm(s), with filtering thresholds chosen for SNPs and short indels based on the sensitivity and specificity on the platinum genomes sample.

Datasets generated with unique Illumina chemistry, and those with different coverages will need to be called in separate subsets. GATK3.3+ UnifiedGenotyper (UG) and GATK3.3+ HaplotypeCaller (HC) will be used for calling SNPs from low and high coverage data, respectively. Multiple samples will be called simultaneously with UnifiedGenotyper, whereas HaplotypeCaller will be applied to individual samples in GVCF mode and subsequently joint genotyping will be carried out with GenotypeGVCFs. We will use the default minimum base qualities for UG and HC, which are currently 17 and 10, respectively. Prior to variant calling UG and HC will both downsample each sample randomly to a maximum coverage of 250. For both of them we don't consider more than 6 alternate alleles (--max\_alternate\_alleles 6). For high coverage data we use calling and emission thresholds of 30 (--standard\_min\_confidence\_threshold\_for\_calling 30 and --standard\_min\_confidence\_threshold\_for\_emitting 30) in accordance with best practices. For low coverage data we use thresholds of 10 and 4 if the sample count is greater than and less than 100, respectively. Indels will be called with a plethora of software packages (see the next section).

If pedigree information is available, then this will be used by UnifiedGenotyper and GenotypeGVCFs in calculation of the InbreedingCoeff annotation, which is used for subsequent variant filtering. Pedigree information will also be used for refinement and phasing. Incomplete pedigrees will be inferred from IBD matrices for sequenced and non-sequenced samples in each cohort.

\subsubsection{Calling of short indels and structural variants}
With UG we will only make a call in the low coverage data, if the number of consensus indels exceeds a threshold of 5 (--min\_indel\_count\_for\_genotyping 5), which is the default value and the value used for phase 1 of 1000G.
%1000G low coverage indel calling by the Broad - "if a candidate indel allele was present in at least 5 reads at a site, it would be passed over to the next step for genotyping, or otherwise it was excluded."
We furthermore use PINDEL, GINDEL, Dindel, samtools, MATE-CLEVER and SOAPdenovo to call indels shorter than 100bp and deletions longer than 100bp. Breakdancer is used to identify structural variants (SVs) in trios. GenomeSTRiP is used for discovery of deletions. We will use best practices of each method to filter variants prior to creating a consensus set. For Pindel for example it will be a requirement that an indel appear in more than 3 samples and with more than 10 supporting reads in total among all samples.
%GoNL Indels (1-20bp) "GATK UG and at least one other algorithm"
%GoNL Deletions (20-100bp) "more than one method" "at least 3 families and transmitted to at least 1 child"
%GoNL Deletions (>100bp) "at least two algorithms" "at least 3 families and transmitted to at least 1 child"

We remove indels with the maximum number of alternate alleles.
%1000G called indels as biallelic.
Similar to 1000G we will remove protein-coding frameshift indels exclusive to low coverage samples and do post-hoc filtering of short indels with a support vector machine (SVM) or random forest (RF) machine learning approach and by applying a MAF threshold of 0.5\%. The latter machine learning method has been used successfully at the Broad Institute. One could classify indels by:
\begin{enumerate}
\item length (continuous or classified)
\item type (homopolymer run (HR) with runs of 6 or more identical nucleotides, tandem repeat (TR))
\item frameshift/nonframeshift in coding regions
\item tandem repeat length (dinucleotide repeat, trinucleotide repeat, STR/microsatellite (2-5/6/9), minisatellite (10-60)
%INSERTION-DELETION VARIANTS IN 179 HUMAN GENOMES - Table 3 - Characteristics of indels
\item HR insertion (more likely), HR deletion (less likely)
\item tandem repeat type (CG, non-CG)
\item SNP at same site, no SNP at same site
\item biallelic, multialleic
\item inside/outside RepeatMasker regions
\item P site / non-P site; i.e. the reference base not N, depth not less than half of average, depth not twice of average, less than 20\% of reads at position have MQ of 0, base not covered. Calculate averages for the autosomes and the X chromosome independently.
\item allele balance
\item strand bias
\item mapping quality
\item number of supporting non-reference reads
\item distance to nearby indels
\end{enumerate}

The NIST NA12878 truth set currently does not hold information on structural variants (SVs). We therefore carry out calling of SVs with multiple callers, apply default calling thresholds and use a consensus dataset as the final SV dataset.

Other variant callers to consider are listed in the table below.
\begin{table}[h]
\centering
\resizebox{\textwidth}{!}{%
\begin{tabular}{|l|l|l|l|l|l|l|l|}
\hline
Name & Year & 1000G & GoNL & Method & de novo assembly & Length & Status \\ \hline
%http://homepages.cwi.nl/~as/software.html
%Mendelian-inheritance-aware discovery and genotyping of midsize and long indels
%.nl
MATE-CLEVER\cite{Marschall2013} & 2014 & & 1-20bp, 20-100bp, \textgreater100bp & read pair / split read  & 1-20bp & \\ \hline
%Pindel good according to ClipCrop paper
%"Pindel only works well for high coverage data, but does not perform well at low coverage." - GINDEL paper
%does Pindel report genotypes?
%http://gmt.genome.wustl.edu/packages/pindel/
Pindel\cite{Ye2009} & 2014 & SV \textgreater50bp & 1-20bp, 20-100bp, /textgreater100 (v0.2.4) & pattern growth? split read &  & 1bp- & Develop \\ \hline
%"Pindel only works well for high coverage data, but does not perform well at low coverage." - GINDEL paper
%"Genome STRiP performs less well for shorter deletions (50–200 bp)" - GINDEL paper
%http://www.engr.uconn.edu/~ywu/
GINDEL\cite{Chu2014GINDEL} & 2014 & No & No &  &  &  \\ \hline
%"Genome STRiP performs less well for shorter deletions (50–200 bp)" - GINDEL paper
%Support for accessing bam files over http, ftp and s3.
%http://www.broadinstitute.org/software/genomestrip/genome-strip
GenomeSTRiP & 2013 & SV \textgreater50bp & 1-20bp, 20-100bp, \textgreater100bp (v1.04.915) & read depth / read pair &  & \\ \hline
% Dindel sensitivity good for low coverage data according to Neuman2012
%https://www.sanger.ac.uk/resources/software/dindel/
Dindel2\cite{Albers01062011} & 2011 & indels & & Bayesian, short-read data & No & \textless50bp \\ \hline
%samtools emits left aligned INDELs
samtools\cite{Li2009} & 2014 & indels &  &  &  & & Develop \\ \hline
FreeBayes & 2014 & Yes &  & Larger INDELs with local assembly &  &  \\ \hline
%Platypus compared to GATK UG, HC and SAMtools
%Platypus emits left aligned INDELs
%evaluated on the high coverage NA12878 trio
%evlauated against fosmid data (ref26)
Platypus\cite{Rimmer2014} & 2014 & indels &  & Local assembly &  & Deletions\textgreater1kb, insertions~300bp & Develop \\ \hline
SOAPIndel/SOAPdenovo & 2014 & & 1-20bp, 20-100bp, \textgreater100bp & de novo assembly of unmapped reads with mapped partner & Yes & \textlessread length & Develop \\ \hline
%Cortex does not require prior mapping
Cortex\cite{Iqbal2012Cortex} & 2013 & Yes &  & reference-free sequence assembly & Yes, Global &  \\ \hline
%GATK emits left aligned INDELs
GATK HaplotypeCaller & 2014 & No & No & Local reassembly & Yes, Local & 1-100bp & Develop \\ \hline
123SV & 2011 & & 20-100bp, \textgreater100bp (v0.9) & read pair &  & \textgreater100bp \\ \hline
DWACSeq & & & \textgreater100bp (v0.7) & read depth based &  & \\ \hline
CNVnator & 2014 & SV \textgreater50bp & \textgreater100bp (v0.2.2) & depth of coverage approach &  & \textgreater window size (e.g. 100) \\ \hline
FACADE & &  & \textgreater100bp & read depth based &  & \\ \hline
%VarScan2.2.2 Not good according to Neuman2012 (figure 4)
%VarScan worse than Dindel (sens and FDR) according to Dindel paper
VarScan\cite{Koboldt01032012} & &  &  &  &  &  \\ \hline
%UG does poorly for low coverage data according to Neuman2012 (figure 4)
GATK UnifiedGenotyper & 2013 &  & 1-20bp, 20-100bp (2.1.8) &  &  & & Abandon  \\ \hline
%BreakDancer Not good according to ClipCrop paper
%Trios only! Used for trios in GoNL paper.
BreakDancer & & SV \textgreater50bp & 20-100bp, \textgreater100bp (v1.1) & Discordant/read pair approach &  & 20-100bp \\ \hline
ClipCrop & 2011 & &  &  &  & & Abandon \\ \hline
DELLY & & SV \textgreater50bp &  &  &  & & \\ \hline
\end{tabular}
}
\caption{Characteristics of INDEL callers. Year refers to latest release/update.}
\label{INDELcallers}
\end{table}

\subsubsection{Calling of chromosomes X and Y and mtDNA}
The X chromosome will be called separately for males and females with different ploidies. The pseudoautosomal regions (PARs) 1 and 2 will however be called like the autosomes; i.e. jointly for males and females with all samples treated as diploid. The Y chromosome will be treated as haploid and variant calling will only be carried out for males. The PARs on the Y chromosome are masked in the reference sequence and not subject to calling. The mitochondrial variants will be called with GATK, VarScan2\cite{Koboldt2012} and MitoSeek and the union set recalled and annotated with GATK prior to filtering. VarScan is chosen, because it performs well at extreme read depths.\cite{Stead2013} MitoSeek is a software package dedicated to calling variants from mtDNA reads. When calling with GATK the ploidy will be set to the mean coverage in the MT contig divided by the mean coverage in the somatic chromosomes.
%http://gatkforums.broadinstitute.org/discussion/1214/can-i-use-gatk-on-non-diploid-organisms
%GoNL - "Consensus sequences were called by GATK."

\begin{table}[h]
\centering
\begin{tabular}{l|l|l|}
\cline{2-3}
\rowcolor[HTML]{FFFFFF} 
                          & Male & Female \\ \hline
\multicolumn{1}{|l|}{X}   & 1    & 2      \\ \hline
\multicolumn{1}{|l|}{Y}   & 1    & N/A    \\ \hline
\multicolumn{1}{|l|}{PAR} & 2    & 2      \\ \hline 
\end{tabular}
\caption{Ploidies used for calling variants on the sex chromosomes.}
\label{table:XYcalling}
\end{table}

\subsection{Variant Filtering}
Filtering of SNPs and short indels will be carried out with GATK3.3+ VariantRecalibrator, which does variant quality score recalibration (VQSR). VQSR will be applied simultaneously across the autosomes and the sex chromosomes despite these having different coverages and some annotations being dependent on read depth. GATK best practices will be used for SNP and INDELs.
%http://gatkforums.broadinstitute.org/discussion/1259/which-training-sets-arguments-should-i-use-for-running-vqsr
For SNPs we will use HapMap III and 1000G phase 1 Omni2.5 sites as truth and training sets (prior probabilities of 15 and 12). High confidence 1000G phase 1 SNPs will be used as an additional training set (prior 10). This dataset does not contain Y chromosome variants. For indels we will use the Mills and Devine and 1000G gold standard as a truth and training set (prior 12). In both cases dbSNP138 or newer will act as a set of known sites.
To build our VQSR Gaussian mixture model we use annotations at each site related to coverage (QD=QualByDepth and DP), strand bias (FS=FisherStrand, SOR=StrandOddsRatio), mapping quality (MQ, MQRankSum, ReadPosRankSum). For indels we use the same annotations, except for MQ being left out.
DP is the approximate read depth after filtering reads with poor mapping quality and bad mates. QD is the variant confidence normalized by the unfiltered depth for the variant allele. FS is a Phred-scaled p-value using Fisher's exact test to detect strand bias. SOR is the odds ratio of a 2x2 contingency table (rows and columns are positive/negative strand and reference/alternate allele) to detect strand bias. MQ is the RMS of the mapping qualities, which serves an an average across reads and samples. MQRankSum is the Z-score from a Wilcoxon rank sum test of alternate vs. reference mapping qualities. ReadPosRankSum is the Z-score from a Wilcoxon rank sum test of alternate vs. reference read position biases.

ROC curves will be generated for sample NA12878, for which SNPs and short INDELs are available for chromosomes 1-22 and X from NIST. The ROC curves will be sorted by the variant quality score log odds ratios calculated by VariantRecalibrator. For structural variants we will use PacBio 54x NA12878 WGS data as a truth set.
%ftp://ftp.1000genomes.ebi.ac.uk/vol1/ftp/technical/working/20131209_na12878_pacbio/si/

For cohorts with 10 or more founder samples the InbreedingCoeff annotation will also be used. It is a likelihood-based Hardy-Weinberg test for the inbreeding among samples. It will be tested by generation of additional ROC curves, whether the inclusion of the InbreedingCoeff annotation worsens the ability to do VQSR correctly, when there are many related samples.

When not doing variant calling with HaplotypeCaller we will also use the annotation HaplotypeScore, which is a statistical measure of more than 2 haplotypes being present at the same site for a sample.

We do a binary heap merge of unfiltered variants to allow two VQSR processes to run simultaneously; i.e. one for SNPs and one for indels.

Unlike 1000G and GoNL we will not filter out multiallelic variants, because we don't find them to be of lower quality than biallelic SNPs and because there is a greater probability of these appearing across thousands of samples from multiple populations from the African continent than in hundreds of samples from one European country. For example 1 in every ~50 SNP in ~2000 Ugandan samples sequenced to a depth of 4x is multiallelic.
%They probably did this, because Beagle3 didn't support multiallelic variants...