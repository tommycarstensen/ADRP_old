\documentclass{article}
\usepackage[utf8]{inputenc}
%\usepackage{longtable} % for tables
\usepackage{graphicx} % for tables
\usepackage[a4paper,left=20mm,right=20mm, top=3cm, bottom=3cm]{geometry}
\usepackage{lscape}
\usepackage{hyperref}
\usepackage{subcaption}
\usepackage{caption}
\usepackage[table,xcdraw]{xcolor} % for tables
\usepackage{multirow} % for tables

\graphicspath{ {figures/} }

\title{African Diversity Reference Panel and H3A Chip Design}
%\author{Deepti Gurdasani \\ \textttt{dg11@sanger.ac.uk} \and Gerrit Botha \\ \textttt{gerrit.botha@uct.ac.za} \and Martin Pollard \\ \textttt{mp15@sanger.ac.uk} \and Tommy Carstensen \\ \textttt{tc9@sanger.ac.uk} \and Cristina Pomilla \\ \textttt{cp8@sanger.ac.uk} \and Ayton Meintjes \\ \textttt{ayton@cbio.uct.ac.za}}
\date{June 10th, 2015}

\begin{document}

\maketitle
%\centerline{\includegraphics[width=80mm]{sang_logo_large}}

\include{sections/1rationale}
\section{Summary of Datasets}
The chip design will incorporate several population groups collated from different parts of Africa. These include samples collated as part of the AGVP, the Genome Diversity in Africa Project (AGVP), the 1000 Genomes project (\href{http://www.1000genomes.org}{http://www.1000genomes.org}), the Simon’s Foundation (\href{http://www.simonsfoundation.org/life-sciences/simons-genome-diversity-project-dataset/}{http://www.simonsfoundation.org/life-sciences/simons-genome-diversity-project-dataset/}), and other shared or publicly available data. These samples have been chosen to be representative of diverse ethno-linguistic and geographical groups across Africa.
\input{tables/samples}

\begin{figure}[h]
\caption{Map showing sample location, sample count and sequencing depth of data to be included.}
\centering
\includegraphics[width=\linewidth]{map_samples}
\end{figure}
\section{Creation of merged reference panel and additional analysis of data}

In addition to the design of the SNP array we intend to use the data to create a merged reference panel as described in section \ref{sec:refine_and_phase}. We will not use IMPUTE2 to create a merged reference panel from existing haplotypes. Instead we will carry out refinement and phasing from the merged filtered variant calls as described in section \ref{sec:curation}, because a larger set of samples will allow more accurate refinement and phasing. The haplotypes will be used for selection of tag SNPs. The further use of this merged reference panel is described below.

In addition to creation and evaluation of the reference panel and design of the SNP array we intend to use the data for additional analyses. Specifically we intend to assess the population structure by calculating principal components, doing cluster analysis and utilising other methods. SNP weights calculated by PCA will also help inform the tag SNP selection process. We intend to calculate allele frequencies and LD patterns, which is necessary for the chip design process. We intend to investigate novelty and do functional annotation of SNPs relative to other sequence datasets and commercial chips, which can aid in the selection of tag SNPs. We intend to asses this across and within populations. The planned analyses are itemized below.

\begin{itemize}
 \item Generate reference panel using all populations irrespective of sample size
 \item Generate SNP array with populations that exceed a sample size threshold of 50
 \item Generate basic descriptive summary statistics for each population before and after genotype refinement; e.g. common and rare SNP count.
 \item Evaluation of imputation accuracy, when using different reference panels and different SNP arrays of comparable size (e.g. designed array, Affymetrix Affy6.0, Illumina HumanOmniExpress and Illumina MEGA array), by imputation into populations not constituting the chip and by removal of the given population imputed into from the haplotype reference panel.
 \item Calculate allelele frequencies and LD patterns
 \begin{itemize}
  \item Allele frequency in each population for biologically relevant variants and all other variants
  \item LD patterns and haplotype lengths within populations
%  \item Haplotype switch error rates, when SNP array data for trios is available - to infer from phasing with different sample sizes - whether increased sample sizes increase phasing accuracy.
 \end{itemize}
 \item Assess population structure
 \begin{itemize}
  \item Principal component analysis
  \item Cluster analysis
  \item {Fixation index ($F_{ST}$) analysis
  % http://biopython.org/wiki/PopGen_Genepop
  % http://vcftools.sourceforge.net/documentation.html#fst
  % https://www.cog-genomics.org/plink2/basic_stats#fst
  }
  \item Calculation of IBD within populations to avoid inclusion of related samples in LD calculations among other things.
  \item Other methods
 \end{itemize}
 \item Functional and structural annotation
  \begin{itemize}
   \item{VEP annotations}
   \item{KEGG pathways}
   \item{Structural interactions within and between proteins
   %http://www.ensembl.org/info/docs/tools/vep/script/vep_options.html#opt_symbol
   %http://www.ensembl.org/info/docs/tools/vep/script/vep_options.html#opt_uniprot
   %http://www.ensembl.org/info/docs/tools/vep/script/vep_options.html#opt_sift
   %http://www.ensembl.org/info/docs/tools/vep/script/vep_options.html#opt_polyphen
   %http://www.ensembl.org/info/docs/tools/vep/script/vep_options.html#opt_everything
   }
   \item{GERP conservation scores}
  \end{itemize}
 \item Novelty and sharing (e.g. $f_2$ variants) relative to sequence datasets and commercial chips
 \begin{itemize}
  \item 1000G phase 1 and/or 3; within Africa and globally
  \item dbSNP138/141+
  \item AGVP
  \item SGDP; within Africa and globally
 \end{itemize}
 \item{Determination of sex chromosome haplogroups
 %Supplementary Figure 18 | mtDNA haplogroups in the GoNL samples
 }
 \item Identification of SNPs in LD with SNPs in the GWAS catalog to aid in tag SNP selection
 \item For each population calculate depth at each site and look for discordance with heterzygous chip sites to infer whether selection of tag SNPs from low depth sites should be avoided.
% \item{Calculate heterozygous chip discordance when calling and refining with different sample sizes to determine whether increased population sizes increase refinement accuracy for African populations}
 \item {Calculate haplotype switch error rates for populations in which trio SNP array data is available to infer - by phasing with different sample sizes - whether increased sample sizes within and across populations increases phasing accuracy for African populations
 %Figure S5. Errors in haplotype estimation.
 }
% \item{Runs of homozygosity for each population
 %Supplementary Figure 11 | Runs of homozygosity (ROH) analysis}
  \item CNV detection for the purpose of inclusion for clinical studies.
  \item Assessment of demographic history.
  \item Assessment of mutational burden in Africa.
  \item Comparison of disease association alleleles within Africa and between the continents.
\end{itemize}


%1000G phase 1 main:
%<http://www.nature.com/nature/journal/v491/n7422/pdf/nature11632.pdf>

%Table 1 | Summary of 1000 Genomes Project phase I data
%Table 2 | Per-individual variant load at conserved sites


%1000G phase 1 supp:
%<http://www.nature.com/nature/journal/v491/n7422/extref/nature11632-s1.pdf>
%p82 Table S1 Low-coverage sequence coverage
%p84 Table S3 Samples with OMNI 2.5M genotypes available, phased using family data available
%p91 Table S10 Cryptic relationships identified by genome-wide SNP analysis (do maxPI_HAT/IBD histogram instead)
%p92 Table S11 Pairwise estimates of FST
%p98 Table S15 Number of variants in linkage disequilibrium (LD) with the SNPs in GWAS catalog
%p101 Figure S3. Sequencing depth and genotyping accuracy
%p103 Figure S5. Errors in haplotype estimation.
%Figure S10. Conservation and variation by sequence annotation and variant type.


%GoNL main:
%<http://www.nature.com/ng/journal/v46/n8/pdf/ng.3021.pdf>
%Figure 1 Discovery of SNVs and structural variation. (a) Venn diagram of all SNVs discovered in GoNL relative to dbSNP (Build 137) and the 1000
%Genomes Project (1000G) Phase 1 and HapMap CEU panels.
%Table 1 Individual variant load of coding mutations


%GoNL supp:
%<http://www.nature.com/ng/journal/v46/n8/extref/ng.3021-S1.pdf>
%Supplementary Figure 2 | Purifying selection per functional annotation and allele frequency
%p7 Supplementary Figure 3 | Distribution of loss-of-function variants per individual genome
%p10 Supplementary Figure 6 | Imputation accuracy using different GWAS chips as imputation panels
%Supplementary Figure 9 | f2 variant sharing between GoNL and 1KG Phase 1 samples
%Supplementary Figure 14 | Phenotype distributions
%Supplementary Figure 18 | mtDNA haplogroups in the GoNL samples
%Supplementary Table 1 | Descriptive statistics of the variants in GoNL
%Supplementary Table 8 | Summary of genotyping chip data of the GoNL samples
%Supplementary Table 9 | SNV quality control metrics throughout the calling pipeline
%Supplementary Table 13 | Variant load per individual
%Supplementary Table 16 | Fst between provinces and regions


%1000G pilot main:
%<http://www.nature.com/nature/journal/v467/n7319/pdf/nature09534.pdf>
%Table 1 | Variants discovered by project, type, population and novelty
%Table 2 | Estimated numbers of potentially functional variants in genes
\section{Generation of haplotypes from raw reads}
Given the need to include large sample sizes from diverse populations for such a resource for development of an African chip, curation of data will involve generation of homogenised data from different datasets, including publicly available data as outlined in Table 1. Given the heterogeneity in sequencing and coverage among datasets (Table 1), these will need to be processed in subsets, and then merged to generate a complete panel of all variants. 

Publicly available curated data, such as reads (bam files) and variant calls (vcf files) from the \href{http://www.1000genomes.org}{1000 Genomes project} and HGDP
%and the \href{http://www.simonsfoundation.org/}{Simon’s Foundation}
will be used as such. For datasets that are sequenced in house, consistent methods will be used for curation.

\subsection{Alignment and preprocessing of reads}
Following generation of raw reads mapping will be carried out to the human reference genome (GRCh37) using the BWA-MEM algorithm of the BWA software package, which is suitable for Illumina reads longer than 70bp.\cite{2013arXiv1303.3997L} Optical and PCR duplicate reads will be marked with Picard MarkDuplicates on a lanelet level. The reads will be sorted by coordinate with samtools sort. The lanelets will be merged to a library level with Picard MergeSamFiles, and duplicate reads will be marked on a library level. The BAMs will be merged to a sample level and then sample level bam improvement will be carried out using GATKv2.8+. This process will consist of a per-sample realignment of reads around known and discovered indels with GATK RealignerTargetCreator and IndelRealigner in addition to base quality score recalibration (BQSR) with GATK BaseRecalibrator and PrintReads.

\subsection{Quality control prior to variant calling}
We require that the percentage of aligned reads is 90\% or greater, which we check with samtools flagstat. Prior to variant calling we check for contamination and sample mix up with VerifyBamID and require that the calculated FREEMIX is less than 0.05.
%Prior to variant calling we check the gender of the samples with GATK3.3+ DepthOfCoverage and require that the ratio between the non-PAR X and Y coverage is less than 2 and greater than 5 for males and females, respectively.

\subsection{Evaluation of variant calling software packages}
The performance of variant callers may be different for SNPs, short indels and SVs. Prior to variant calling across all datasets we determined the best variant calling and filtering method for SNPs for African low coverage data. Specifically variant calling was carried out for chromosome 20 of 1,986 Ugandan samples sequenced to an average 4x coverage on Illumina HiSeq 2000. We used several algorithms; samtools, GATK HaplotypeCaller (HC) and GATK UnifiedGenotyper (UG). To make assesment of the sensitivity and specificity for each variant caller possible calling was carried out with a sample from the \href{http://genomeinabottle.org}{Genome in a Bottle (GiaB)} highly curated set sequenced to a similar coverage (6x). The GiaB sample (NA12878) represents a CEU sample from a 12 person pedigree in 1000G that has gone through extensive curation and validation of variants.\cite{Zook2014}
We calculated the sensitivity and specificity of calls relative to the highly curated variant sites for the NA12878 sample, to identify the caller with greatest area under ROC curve at different filtering thresholds (Figure \ref{fig:roc}). We used two different filtering thresholds for this analysis: the SNP quality metric (QUAL), and the VQSLOD score obtained using the VQSR model implemented by GATK for different callers. Variant Quality score recalibration (VQSR) based filtering approaches seem to perform better than filtering only on variant quality for most calls. With this evaluation, we show that UnifiedGenotyper3.2 shows the best area under ROC curve with the lowest FDR for a given sensitivity for SNPs and indels (Figure \ref{fig:roc}).% All callers, however produce very low sensitivity and high FDRs for indel calls (Figure \ref{fig:roc_indels}).

%These callers may be different for SNPs, short indels and SVs. Calling and filtering of SNPs, indels and long deletions will be carried out separately using the chosen algorithm(s), with filtering thresholds chosen for SNPs and short indels based on the sensitivity and specificity on the platinum genomes sample.
%However, further exploration of filtering approaches for indel calls is needed, including appropriate normalisation of model training sets, as this could potentially improve the sensitivity for a given FDR. Additionally, a new release of HaplotypeCaller corrects issues with previous releases, potentially greatly improving the sensitivity for SNPs and indels. A comparison of these callers with consistent filtering methods will inform the best method to use for calling SNPs and indels within low coverage (4x) data. Choosing appropriate callers and filtering methods is crucial maximising variant discovery while maintaining low false discovery on the panel curated for the development of the chip array.

\begin{figure}[h]
\captionsetup{width=0.8\textwidth}
\caption{An evaluation of calling algorithms for SNPs and indels in low coverage data. The figure depicts a comparison of various calling algorithms using different filters for calling SNPs (left) and indels (right) in low coverage data. The x axis represents the false discovery rate (FDR), which is defined as the proportion of calls produced by a given algorithm that are false positives at a given filtering threshold. The y axis represents the true positive rate or the sensitivity, which is the proportion of all true calls in the GiaB sample that are captured by a given algorithm for a given filtering threshold. The curves are generated by varying filtering thresholds for each algorithm. UG: UnifiedGenotyper; FB: FreeBayes; HC: HaplotypeCaller; QUAL: Phred-scaled quality score; VQSLOD: Variant Quality Recalibration scores; NIST: National Institute of Standards and Technology.}
\label{fig:roc}
\centering
    \begin{subfigure}[b]{0.45\textwidth}
        \includegraphics[width=\textwidth]{FDR_snps}
        \caption{SNPs}
        \label{fig:roc_snps}
    \end{subfigure}%
    \begin{subfigure}[b]{0.45\textwidth}
        \includegraphics[width=\textwidth]{FDR_indels}
        \caption{Indels}
        \label{fig:roc_indels}
    \end{subfigure}%
\end{figure}

For high coverage data, we evaluated the NA12878 sample alone re-sequenced on the X-10 pipeline with the highly curated set of variants produced by GiaB.1 We showed that the X-10 and the new pipeline used with it produced high quality data that when normalised showed a high degree of sensitivity versus the GiaB reference set and a low false discovery rate (Table \ref{table:highcoverage}). When the Illumina 50x Platinum set, sequenced on the Illumina Hiseq 2000 platform was reprocessed through the same pipeline it produced similarly high results, though the INDEL sensitivity and false discovery rate were slightly better. We believe this may be because of the higher coverage and PCR free library preparation used in the creation of this sequence.

\begin{table}[h]
\centering
%\resizebox{\textwidth}{!}{%
\begin{tabular}{l|lrrrrr}
\rowcolor[HTML]{333333} 
{\color[HTML]{FFFFFF} Sample} & {\color[HTML]{FFFFFF} Type} & {\color[HTML]{FFFFFF} TP} & {\color[HTML]{FFFFFF} FP} & {\color[HTML]{FFFFFF} FN} & {\color[HTML]{FFFFFF} Sensitivity} & {\color[HTML]{FFFFFF} FDR} \\
 & \cellcolor[HTML]{C0C0C0}SNP & \cellcolor[HTML]{C0C0C0}2914294 & \cellcolor[HTML]{C0C0C0}28213 & \cellcolor[HTML]{C0C0C0}10266 & \cellcolor[HTML]{C0C0C0}0.9965 & \cellcolor[HTML]{C0C0C0}0.0096 \\
\multirow{-2}{*}{Replicate 1} & INDEL & 413639 & 23145 & 34743 & 0.9255 & 0.0530 \\
 & \cellcolor[HTML]{C0C0C0}SNP & \cellcolor[HTML]{C0C0C0}2908260 & \cellcolor[HTML]{C0C0C0}26311 & \cellcolor[HTML]{C0C0C0}14452 & \cellcolor[HTML]{C0C0C0}0.9951 & \cellcolor[HTML]{C0C0C0}0.0090 \\
\multirow{-2}{*}{Replicate 2} & INDEL & 393431 & 24110 & 52542 & 0.8822 & 0.0577 \\
 & \cellcolor[HTML]{C0C0C0}SNP & \cellcolor[HTML]{C0C0C0}2917290 & \cellcolor[HTML]{C0C0C0}21400 & \cellcolor[HTML]{C0C0C0}9844 & \cellcolor[HTML]{C0C0C0}0.9966 & \cellcolor[HTML]{C0C0C0}0.0073 \\
\multirow{-2}{*}{Platinum} & INDEL & 437875 & 8802 & 2297 & 0.9948 & 0.0197
\end{tabular}
%}
\caption{Intersect of NA12878 samples with GiaB reference calls version 0.2.}
\label{table:FDRhigh}
\end{table}

\input{sections/curation/calling}

\input{sections/curation/filtering}

\input{sections/curation/merging}

\subsection{Genotype refinement and phasing}
We carry out genotype refinement and phasing with SHAPEIT2.\cite{Delaneau2012} Initial refinement of genotype likelihoods is however carried out with Beagle4.\cite{Browning20071084} The posterior probabilities calculated by Beagle4 are then used as input for SHAPEIT2 along with a haplotype scaffold generated from SNP array data available for the same populations. The SNP array data undergoes QC per population and is phased with SHAPEIT2 across all populations to generate the haplotype scaffold. MVNcall can also utilize a haplotype scaffold and unlike SHAPEIT2 works for multiallelic sites.\cite{Menelaou2013} However, we use SHAPEIT2, because it improves on the accuracy of phasing.\cite{2014Delaneau} The Illumina Omni2.5M SNP array has been shown to be an optimal/sufficient haplotype scaffold size in African populations.\cite{Menelaou2013}\cite{2014Delaneau} Pedigree information will be used by SHAPEIT2 when available. For Beagle4 this is less important, as this is only an initial refinement used as input for SHAPEIT2. We use the duoHMM method of SHAPEIT2 for phasing, because it has been shown to have a lower switch error rate, when pedigree information is available.\cite{OConnell2014}
%Figure 8: Overall percent genotype discordance for different scaffold SNP density. Results are presented by population groups: African, European and Asian.
%Is MVNCall only better, if BEAGLE doesn't have phasing information?
%2014Delaneau - "It has been observed that the Beagle method does not have this property, and that Thunder and Impute2 benefit from using an initial set of haplotypes estimated via Beagle."
%2014Delaneu - "This approach generalizes our MVNcall, approach which is designed to phase one variant site at a time onto a haplotype scaffold, and improves upon its accuracy, by phasing multiple sites jointly onto the scaffold and using a more sophisticated underlying model."

%% https://mathgen.stats.ox.ac.uk/genetics_software/shapeit/shapeit.html#gcall
\include{sections/4tagSNPselection}
\section{Validation of selected tag SNPs}

As sequence data, particularly low coverage short read data has been prone to sequencing errors, we will validate all tagging sites based on a large data repository of validated variant sites collated by the vendor. Novel variant sites will be further validated by the vendor, prior to chip design, and proxies of sites that cannot be covered will be investigated and filled in for generation of the final array.

We will check that no SNPs are within 10bp of each other. To ensure good coverage across the genome we will also check that no intervals outside of the centromere regions without tag SNPs in them are longer than 5kbp.
%\section{Collaborators}

\begin{table}[h]
\centering
\resizebox{\textwidth}{!}{%
\begin{tabular}{llll}
\hline
\textbf{Name}       & \textbf{Affiliation}                                     & \textbf{Role}          & \textbf{E-mail}              \\
\hline
Cristina Pomilla    & Wellcome Trust Sanger Institute                          & Project Management     & cp8@sanger.ac.uk             \\
Deepti Gurdasani    & Wellcome Trust Sanger Institute                          & Analyst                & dg11@sanger.ac.uk            \\
Martin Pollard      & Wellcome Trust Sanger Institute, University of Cambridge & Analyst                & mp15@sanger.ac.uk            \\
Tommy Carstensen    & Wellcome Trust Sanger Institute, University of Cambridge & Analyst                & tc9@sanger.ac.uk             \\
Luca Pagani         & Wellcome Trust Sanger Institute                          & Analyst                & lp8@sanger.ac.uk             \\
Ayton Meintjes      & University of Cape Town                                  & Analyst                & ayton@cbio.uct.ac.za         \\
Emile Chimusa       & University of Cape Town                                  & Analyst                & echimusa@gmail.com           \\
Gerrit Botha        & University of Cape Town                                  & Analyst                & gerrit.botha@uct.ac.za       \\
Martin Sikora       & University of Copenhagen                                 &                        & martin.sikora@snm.ku.dk      \\
Laura Botigue       & Stony Brook University                                   &                        & laura.botigue@stonybrook.edu \\
Audrey Duncanson    & Wellcome Trust                                           & H3A Project Management & a.duncanson@wellcome.ac.uk   \\
Ken Wiley           & National Institute of Health                             & H3A Project Management & wileykl@mail.nih.gov         \\
Zane Lombard        & University of Witwatersrand                              &                        & Zane.Lombard@wits.ac.za      \\
Erik Garrison       & Wellcome Trust Sanger Institute                          & Scientific Advisor     & eg10@sanger.ac.uk            \\
Shane McCarthy      & Wellcome Trust Sanger Institute                          & Scientific Advisor     & sm15@sanger.ac.uk            \\
Christiane Hertz-Fowler & University of Liverpool & & C.Hertz-Fowler@liverpool.ac.uk \\
Adebowale Adeyemo & National Institute of Health & & adeyemoa@mail.nih.gov \\
Brenna Henn         & Stony Brook University                                   & Principal Investigator & brenna.henn@stonybrook.edu   \\
Nicola Mulder       & University of Cape Town                                  & Principal Investigator & nicola.mulder@uct.ac.za      \\
Chris-Tyler Smith   & Wellcome Trust Sanger Institute                          & Principal Investigator & cts@sanger.ac.uk             \\
Manjinder S. Sandhu & Wellcome Trust Sanger Institute, University of Cambridge & Principal Investigator & ms23@sanger.ac.uk           \\
\hline
\end{tabular}
}
\end{table}

\bibliographystyle{unsrt}%Used BibTeX style is unsrt
\bibliography{references/references}

%We don't filter based on the percentage of mapped reads. Instead we filter based on heterozygosity. The two parameters are related.

%“Version 4 has multiple improvements: support for multi-allelic markers” – BEAGLE4 site

%An improved method for calling genotypes and phasing low-coverage data when one also has chip data is to phase the chip data to make a haplotype scaffold and then phase the sequence data onto the scaffold. Both steps are carried out with SHAPEIT2.

\end{document}
