\documentclass{article}
\usepackage[utf8]{inputenc}
\usepackage{longtable} % for tables
\usepackage{graphicx} % for tables
\usepackage[a4paper,left=20mm,right=20mm, top=3cm, bottom=3cm]{geometry}
\usepackage{lscape}
\usepackage{hyperref}
\usepackage{subcaption}
\usepackage{caption}
\usepackage[table,xcdraw]{xcolor} % for tables
\usepackage{multirow} % for tables

\graphicspath{ {figures/} }

\title{African Genome Diversity Reference Panel and Chip Design}
\author{Tommy Carstensen, Deepti Gurdasani, Martin Pollard, Cristina Pomilla}
\date{January 2015}

\begin{document}

\maketitle
\centerline{\includegraphics[width=80mm]{sang_logo_large}}

\include{sections/1rationale}
\section{Summary of Datasets}
The chip design will incorporate several population groups collated from different parts of Africa. These include samples collated as part of the AGVP, the Genome Diversity in Africa Project (AGVP), the 1000 Genomes project (\href{http://www.1000genomes.org}{http://www.1000genomes.org}), the Simon’s Foundation (\href{http://www.simonsfoundation.org/life-sciences/simons-genome-diversity-project-dataset/}{http://www.simonsfoundation.org/life-sciences/simons-genome-diversity-project-dataset/}), and other shared or publicly available data. These samples have been chosen to be representative of diverse ethno-linguistic and geographical groups across Africa.
\input{tables/samples}

\begin{figure}[h]
\caption{Map showing sample location, sample count and sequencing depth of data to be included.}
\centering
\includegraphics[width=\linewidth]{map_samples}
\end{figure}
\section{Generation of haplotypes from raw reads}
Given the need to include large sample sizes from diverse populations for such a resource for development of an African chip, curation of data will involve generation of homogenised data from different datasets, including publicly available data as outlined in Table 1. Given the heterogeneity in sequencing and coverage among datasets (Table 1), these will need to be processed in subsets, and then merged to generate a complete panel of all variants. 

Publicly available curated data, such as reads (bam files) and variant calls (vcf files) from the \href{http://www.1000genomes.org}{1000 Genomes project} and HGDP
%and the \href{http://www.simonsfoundation.org/}{Simon’s Foundation}
will be used as such. For datasets that are sequenced in house, consistent methods will be used for curation.

\subsection{Alignment and preprocessing of reads}
Following generation of raw reads mapping will be carried out to the human reference genome (GRCh37) using the BWA-MEM algorithm of the BWA software package, which is suitable for Illumina reads longer than 70bp.\cite{2013arXiv1303.3997L} Optical and PCR duplicate reads will be marked with Picard MarkDuplicates on a lanelet level. The reads will be sorted by coordinate with samtools sort. The lanelets will be merged to a library level with Picard MergeSamFiles, and duplicate reads will be marked on a library level. The BAMs will be merged to a sample level and then sample level bam improvement will be carried out using GATKv2.8+. This process will consist of a per-sample realignment of reads around known and discovered indels with GATK RealignerTargetCreator and IndelRealigner in addition to base quality score recalibration (BQSR) with GATK BaseRecalibrator and PrintReads.

\subsection{Quality control prior to variant calling}
We require that the percentage of aligned reads is 90\% or greater, which we check with samtools flagstat. Prior to variant calling we check for contamination and sample mix up with VerifyBamID and require that the calculated FREEMIX is less than 0.05.
%Prior to variant calling we check the gender of the samples with GATK3.3+ DepthOfCoverage and require that the ratio between the non-PAR X and Y coverage is less than 2 and greater than 5 for males and females, respectively.

\subsection{Evaluation of variant calling software packages}
The performance of variant callers may be different for SNPs, short indels and SVs. Prior to variant calling across all datasets we determined the best variant calling and filtering method for SNPs for African low coverage data. Specifically variant calling was carried out for chromosome 20 of 1,986 Ugandan samples sequenced to an average 4x coverage on Illumina HiSeq 2000. We used several algorithms; samtools, GATK HaplotypeCaller (HC) and GATK UnifiedGenotyper (UG). To make assesment of the sensitivity and specificity for each variant caller possible calling was carried out with a sample from the \href{http://genomeinabottle.org}{Genome in a Bottle (GiaB)} highly curated set sequenced to a similar coverage (6x). The GiaB sample (NA12878) represents a CEU sample from a 12 person pedigree in 1000G that has gone through extensive curation and validation of variants.\cite{Zook2014}
We calculated the sensitivity and specificity of calls relative to the highly curated variant sites for the NA12878 sample, to identify the caller with greatest area under ROC curve at different filtering thresholds (Figure \ref{fig:roc}). We used two different filtering thresholds for this analysis: the SNP quality metric (QUAL), and the VQSLOD score obtained using the VQSR model implemented by GATK for different callers. Variant Quality score recalibration (VQSR) based filtering approaches seem to perform better than filtering only on variant quality for most calls. With this evaluation, we show that UnifiedGenotyper3.2 shows the best area under ROC curve with the lowest FDR for a given sensitivity for SNPs and indels (Figure \ref{fig:roc}).% All callers, however produce very low sensitivity and high FDRs for indel calls (Figure \ref{fig:roc_indels}).

%These callers may be different for SNPs, short indels and SVs. Calling and filtering of SNPs, indels and long deletions will be carried out separately using the chosen algorithm(s), with filtering thresholds chosen for SNPs and short indels based on the sensitivity and specificity on the platinum genomes sample.
%However, further exploration of filtering approaches for indel calls is needed, including appropriate normalisation of model training sets, as this could potentially improve the sensitivity for a given FDR. Additionally, a new release of HaplotypeCaller corrects issues with previous releases, potentially greatly improving the sensitivity for SNPs and indels. A comparison of these callers with consistent filtering methods will inform the best method to use for calling SNPs and indels within low coverage (4x) data. Choosing appropriate callers and filtering methods is crucial maximising variant discovery while maintaining low false discovery on the panel curated for the development of the chip array.

\begin{figure}[h]
\captionsetup{width=0.8\textwidth}
\caption{An evaluation of calling algorithms for SNPs and indels in low coverage data. The figure depicts a comparison of various calling algorithms using different filters for calling SNPs (left) and indels (right) in low coverage data. The x axis represents the false discovery rate (FDR), which is defined as the proportion of calls produced by a given algorithm that are false positives at a given filtering threshold. The y axis represents the true positive rate or the sensitivity, which is the proportion of all true calls in the GiaB sample that are captured by a given algorithm for a given filtering threshold. The curves are generated by varying filtering thresholds for each algorithm. UG: UnifiedGenotyper; FB: FreeBayes; HC: HaplotypeCaller; QUAL: Phred-scaled quality score; VQSLOD: Variant Quality Recalibration scores; NIST: National Institute of Standards and Technology.}
\label{fig:roc}
\centering
    \begin{subfigure}[b]{0.45\textwidth}
        \includegraphics[width=\textwidth]{FDR_snps}
        \caption{SNPs}
        \label{fig:roc_snps}
    \end{subfigure}%
    \begin{subfigure}[b]{0.45\textwidth}
        \includegraphics[width=\textwidth]{FDR_indels}
        \caption{Indels}
        \label{fig:roc_indels}
    \end{subfigure}%
\end{figure}

For high coverage data, we evaluated the NA12878 sample alone re-sequenced on the X-10 pipeline with the highly curated set of variants produced by GiaB.1 We showed that the X-10 and the new pipeline used with it produced high quality data that when normalised showed a high degree of sensitivity versus the GiaB reference set and a low false discovery rate (Table \ref{table:highcoverage}). When the Illumina 50x Platinum set, sequenced on the Illumina Hiseq 2000 platform was reprocessed through the same pipeline it produced similarly high results, though the INDEL sensitivity and false discovery rate were slightly better. We believe this may be because of the higher coverage and PCR free library preparation used in the creation of this sequence.

\begin{table}[h]
\centering
%\resizebox{\textwidth}{!}{%
\begin{tabular}{l|lrrrrr}
\rowcolor[HTML]{333333} 
{\color[HTML]{FFFFFF} Sample} & {\color[HTML]{FFFFFF} Type} & {\color[HTML]{FFFFFF} TP} & {\color[HTML]{FFFFFF} FP} & {\color[HTML]{FFFFFF} FN} & {\color[HTML]{FFFFFF} Sensitivity} & {\color[HTML]{FFFFFF} FDR} \\
 & \cellcolor[HTML]{C0C0C0}SNP & \cellcolor[HTML]{C0C0C0}2914294 & \cellcolor[HTML]{C0C0C0}28213 & \cellcolor[HTML]{C0C0C0}10266 & \cellcolor[HTML]{C0C0C0}0.9965 & \cellcolor[HTML]{C0C0C0}0.0096 \\
\multirow{-2}{*}{Replicate 1} & INDEL & 413639 & 23145 & 34743 & 0.9255 & 0.0530 \\
 & \cellcolor[HTML]{C0C0C0}SNP & \cellcolor[HTML]{C0C0C0}2908260 & \cellcolor[HTML]{C0C0C0}26311 & \cellcolor[HTML]{C0C0C0}14452 & \cellcolor[HTML]{C0C0C0}0.9951 & \cellcolor[HTML]{C0C0C0}0.0090 \\
\multirow{-2}{*}{Replicate 2} & INDEL & 393431 & 24110 & 52542 & 0.8822 & 0.0577 \\
 & \cellcolor[HTML]{C0C0C0}SNP & \cellcolor[HTML]{C0C0C0}2917290 & \cellcolor[HTML]{C0C0C0}21400 & \cellcolor[HTML]{C0C0C0}9844 & \cellcolor[HTML]{C0C0C0}0.9966 & \cellcolor[HTML]{C0C0C0}0.0073 \\
\multirow{-2}{*}{Platinum} & INDEL & 437875 & 8802 & 2297 & 0.9948 & 0.0197
\end{tabular}
%}
\caption{Intersect of NA12878 samples with GiaB reference calls version 0.2.}
\label{table:FDRhigh}
\end{table}

\input{sections/curation/calling}

\input{sections/curation/filtering}

\input{sections/curation/merging}

\subsection{Genotype refinement and phasing}
We carry out genotype refinement and phasing with SHAPEIT2.\cite{Delaneau2012} Initial refinement of genotype likelihoods is however carried out with Beagle4.\cite{Browning20071084} The posterior probabilities calculated by Beagle4 are then used as input for SHAPEIT2 along with a haplotype scaffold generated from SNP array data available for the same populations. The SNP array data undergoes QC per population and is phased with SHAPEIT2 across all populations to generate the haplotype scaffold. MVNcall can also utilize a haplotype scaffold and unlike SHAPEIT2 works for multiallelic sites.\cite{Menelaou2013} However, we use SHAPEIT2, because it improves on the accuracy of phasing.\cite{2014Delaneau} The Illumina Omni2.5M SNP array has been shown to be an optimal/sufficient haplotype scaffold size in African populations.\cite{Menelaou2013}\cite{2014Delaneau} Pedigree information will be used by SHAPEIT2 when available. For Beagle4 this is less important, as this is only an initial refinement used as input for SHAPEIT2. We use the duoHMM method of SHAPEIT2 for phasing, because it has been shown to have a lower switch error rate, when pedigree information is available.\cite{OConnell2014}
%Figure 8: Overall percent genotype discordance for different scaffold SNP density. Results are presented by population groups: African, European and Asian.
%Is MVNCall only better, if BEAGLE doesn't have phasing information?
%2014Delaneau - "It has been observed that the Beagle method does not have this property, and that Thunder and Impute2 benefit from using an initial set of haplotypes estimated via Beagle."
%2014Delaneu - "This approach generalizes our MVNcall, approach which is designed to phase one variant site at a time onto a haplotype scaffold, and improves upon its accuracy, by phasing multiple sites jointly onto the scaffold and using a more sophisticated underlying model."

%% https://mathgen.stats.ox.ac.uk/genetics_software/shapeit/shapeit.html#gcall
%\subsection{Genotype refinement and phasing}
Initial refinement of genotype likelihoods is carried out with Beagle4. Chip data available for the same populations, which has undergone QC, will be phased with SHAPEIT2 across all populations. This will provide a haplotype scaffold for MVNcall, which will be used for the final refinement. 2.5M SNPs has been shown to be an optimal/sufficient haplotype scaffold size in African populations.\cite{Menelaou2013} Pedigree information will be used by Beagle4 and SHAPEIT2, when available. Prior to refinement with Beagle4 we recode haploid male genotypes (non-PAR X and Y) to homozygous diploid genotypes. We use the duoHMM method of SHAPEIT2 for phasing, because it has been shown to have a lower switch error rate, when pedigree information is available.\cite{OConnell2014}
%Figure 8: Overall percent genotype discordance for different scaffold SNP density. Results are presented by population groups: African, European and Asian.
%Is MVNCall only better, if BEAGLE doesn't have phasing information?
%\section{Evaluation of Variant Callers}
\subsection{Introduction}
We need to generate a set of high quality variants to be used in a reference panel and to be used as a set of candidate SNPs for tagSNP selection. Therefore it important to evaluate the various methods for calling variants such as SNPs, short indels and structural variants (SVs) for the sequence reads.
\subsection{Methods}
Variants ...
\section{Merging of Datasets}

Following curation of individual datasets, these will need to be merged to homogenise variant calls across all data, and generate a square matrix of variant calls, where the union of variants across all samples is available for each population. In order to do this, a union set of calls will be generated for variants that have passed filtering in each dataset (Figure 1). These sites will then be recalled in each dataset to generate genotype likelihoods at these sites. Following this, genotype refinement will be carried out using Beagle v4. For data with pedigree information, this information will be utilised in calling as well as phasing of data. Following refinement of genotype probabilities in Beagle, haplotype phasing will be carried out with SHAPEIT2 to maximise phasing accuracy, especially given the relatedness among some sample sets. MVNcall\cite{Menelaou2013} will then be applied where genotype information is available, within and outside pedigrees to further refine genotype calls based on generating a haplotype scaffold from the 2.5M Omni genotype data available on many sample sets. A possible workflow parallelising data curation across institutions is detailed in Appendix 1. 

\begin{figure}[h]
\caption{Homogenised calling across all datasets to generate a single panel}
\centering
\includegraphics[width=0.5\textwidth]{calling}
\end{figure}

\section{Identification of Informative Variants for the Chip Array}

For identification of informative variants on the chip array, we will utilise a hybrid algorithm with cycles of LD based pairwise tagging, and imputation, as has been previously described.\cite{} Only populations with at least 50 samples will be included in the chip design process, as pairwise LD evaluation would be difficult in smaller samples. Here, the term ‘population’ should be considered distinct from an ethno-linguistic group/project, and constitutes a group of individuals/samples that appear genetically homogeneous, without any significant substructure or clustering. In order to account for different sample sizes and LD differentiation among populations, we will carry out multi-population pairwise tagging, as described below. We seek only to tag common SNPs (MAF\textgreater5\%) for generation of this array.

We have developed a multi-population tagging algorithm based on the algorithm TAGster for WGS data.\cite{Xu2007} The methods we used for tagging were identical to those used by TAGster; however, by using seeking and indexing approaches we were able to optimise the computational efficiency of the algorithm by an order of magnitude (unpublished data, Carstensen et al.). We briefly outline the tagging algorithm as follows (Figure 2):

\begin{enumerate}
\item Calculate LD (\textit{r}\textsuperscript{2}) between each SNP and all other SNPs in the flanking 250 KB region for each population separately. MAF thresholds are imposed at this stage, and only pairs of SNPs where both exceed the MAF threshold are included.
\item For each SNP not already in the tagging set, a count of SNPs in the target set that are in LD exceeding a given threshold \textit{r}\textsuperscript{2} with it is generated across the genome and summed across all populations.
\item The most informative SNP (the SNP with most target SNPs in LD with it summed across population) is chosen as the tagging SNP and added to the set of tagging SNPs.
\item This tagging SNP and SNPs in LD with it are now removed from the set of target SNPs. This process is carried out separately for each population, so that a separate set of target SNPs is maintained for each population set. However, SNPs in LD with the tagging SNP can still be picked up as tagging SNPs themselves if they independently tag the maximum no. of SNPs in any iteration.
Steps 2-3 are repeated until either a specified number of SNPs or all target SNPs (chosen as SNPs above a specific MAF threshold per-population) are tagged across all population sets, or until a specific number of SNPs is reached, as specified.
\end{enumerate}

\begin{figure}[h]
\caption{Hybrid tagging and imputation algorithm for chip design.}
\centering
\includegraphics[width=0.8\textwidth]{tagSNPselection}
\end{figure}

Although this method works well across populations, it only carries out pairwise single-marker tagging. Haplotype based, or multi-marker tagging would potentially be more efficient, and select fewer tagging sites. In order to incorporate haplotype based tagging into our model, we use a hybrid method, with cycles of tagging and imputation, as has been described before.2 

In order to implement this method, in the first cycle, we select a maximum number of pre-defined tagging SNPs. Following this, these tagging SNPs are selected from each population, and imputation is carried out using a reference panel, to identify additional sites in each population that are tagged at an $r^{2}$ threshold above 0.80 by the tagging sites. These sites are removed from the target set for each population, and do not contribute to the next cycle, thereby making the process more efficient. For imputation, we include the entire reference panel curated from all samples in table \ref{table:samples}, including the 1000 Genomes multi-ethnic samples from Europe, Asian and the Americas, to maximise imputation accuracy. However, for imputation into each population, samples from the given population are removed from the reference set, so the imputation panel does not include any samples from the population being evaluated for tagging. This ‘leave one population out’ approach would produce relatively conservative results for tagging, with more variants being tagged than if a subsample of the population was included in the reference panel. This is a more realistic scenario, as it is not necessary that any given population genotyped on the chip in future would be represented in the reference panel.

Additionally, pre-selected known biologically relevant variants, valuable to the studies planned for consortia can be included on the array, to replace certain tag SNPs. We have previously shown that a 1M tagging variants chosen using the described algorithm can produce \textgreater80\% coverage across diverse populations in Africa. Based on this, we plan to carry out approximately 10 cycles to capture 1.2M tagging variants, in order to prioritise variants to include in the design of a 1M chip array. Following this, we will further validate our tagging algorithm among populations with smaller sample sizes that were not included in the development in the chip array, by selecting tagging variants among these and imputing with the reference panel, excluding these populations. We estimate coverage of each population by such a chip array using imputation. Coverage is defined as the proportion of common variation captured at an r2>0.80 across the genome in a given population with the combined reference panel (excluding the population being evaluated). $r^{2}$, here, is calculated as the correlation between the sequence data and imputed data on a hypothetical 1M chip array for common variation. 

\include{sections/7variantcaller_evaluation}
\include{sections/8validation}

\bibliographystyle{unsrt}%Used BibTeX style is unsrt
\bibliography{references/references}

%We don't filter based on the percentage of mapped reads. Instead we filter based on heterozygosity. The two parameters are related.

%“Version 4 has multiple improvements: support for multi-allelic markers” – BEAGLE4 site

%An improved method for calling genotypes and phasing low-coverage data when one also has chip data is to phase the chip data to make a haplotype scaffold and then phase the sequence data onto the scaffold. Both steps are carried out with SHAPEIT2.

\end{document}
